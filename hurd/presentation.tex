\documentclass[10pt]{beamer}

\usepackage[brazilian]{babel}
\usepackage[utf8]{inputenc}
\usepackage{graphicx}
\usepackage{mathtools}
\usepackage{amsthm}
\usepackage{thmtools,thm-restate}
\usepackage{amsfonts}
\usepackage{hyperref}
\usepackage[singlelinecheck=false]{caption}
\usepackage[backend=biber,url=true,doi=true,eprint=false,style=alphabetic]{biblatex}
\usepackage[justification=centering]{caption}
\usepackage{indentfirst}
\usepackage{enumitem}
\usepackage{algorithm}
\usepackage{algpseudocode}
\usepackage{listings}
% Timeline
\usepackage{xcolor}
\newcommand\ytl[2]{
\parbox[b]{8em}{\hfill{\color{cyan}\bfseries\sffamily #1}~$\cdots\cdots$~}\makebox[0pt][c]{$\bullet$}\vrule\quad \parbox[c]{7.5cm}{\vspace{2pt}\color{red!40!black!80}\raggedright\sffamily #2.\\[7pt]}\\[-3pt]}

% Fix beamer and enumitem conflicts.
\setitemize{label=\usebeamerfont*{itemize item}%
  \usebeamercolor[fg]{itemize item}
\usebeamertemplate{itemize item}}

% Remove line breaks.
\setbeamertemplate{bibliography entry title}{}
\setbeamertemplate{bibliography entry location}{}
\setbeamertemplate{bibliography entry note}{}

\uselanguage{Brazilian}
\languagepath{Brazilian}
\usetheme{Berlin}

\addbibresource{references.bib}

\newcommand\nmfootnote[1]{%
  \begingroup
  \renewcommand\thefootnote{}\footnote{#1}%
  \addtocounter{footnote}{-1}%
  \endgroup
}

\makeatletter
\def\subsection{\@startsection{subsection}{3}%
  \z@{.5\linespacing\@plus.7\linespacing}{.1\linespacing}%
  {\normalfont}}
\makeatother

\DeclareMathOperator*{\argmin}{arg\,min}
\DeclareMathOperator*{\argmax}{arg\,max}

\newcommand\defeq{\mathrel{\overset{\makebox[0pt]{\mbox{\normalfont\tiny\sffamily def}}}{=}}}

\floatname{algorithm}{Algoritmo}
\algrenewcommand\algorithmicrequire{\textbf{Input}}
\algrenewcommand\algorithmicensure{\textbf{Output}}

\captionsetup[table]{labelsep=space}

\theoremstyle{plain}

\newtheorem{proposition}{Proposição}
\newtheorem{exercise}{Exercício}

\newcommand{\set}[1]{\mathbf{#1}}
\newcommand{\pr}{\mathbb{P}}
\renewcommand{\implies}{\Rightarrow}

\newcommand{\bigo}{\mathcal{O}}
\newcommand{\p}{\pause}

\setlength{\parskip}{1em}

\lstset{frameround=fttt,
  language=[5.3]Lua,
  numbers=left,
  breaklines=true,
  keywordstyle=\bfseries,
  basicstyle=\ttfamily,
}

\newcommand\Fontsmall{\fontsize{12}{7.2}\selectfont}

\newcommand{\code}[1]{\lstinline[mathescape=true]{#1}}
\newcommand{\mcode}[1]{\lstinline[mathescape]!#1!}

\title{GNU Hurd}
\subtitle{MAC0422 --- Estudo de Caso}
\author{Renato Lui Geh}

\begin{document}

\frame{\titlepage}

\begin{frame}
  \frametitle{Índice}
  \tableofcontents
\end{frame}

%--------------------------------------------------------------------------------------------------

\section{História}

\begin{frame}
  \frametitle{História do GNU/Hurd}

  \begin{table}
    \begin{minipage}[t]{\linewidth}
      \color{gray}
      \rule{\linewidth}{1pt}
      \ytl{1983}{Richard Stallman (RMS) cria o projeto GNU}
      \ytl{1986}{RMS decide usar o TRIX como kernel}
      \ytl{1988}{É decidido usar o Mach como kernel}
      \ytl{1991}{GNU Hurd é anunciado ao público}
      \ytl{1994}{Primeiro boot}
      \ytl{1994}{Emacs e gcc rodam pela primeira vez}
      \ytl{1995}{ext2fs, ftp}
      \ytl{1996}{NFS e GNU Hurd 0.1}
      \ytl{1997}{GNU Hurd 0.2}
      \bigskip
      \rule{\linewidth}{1pt}
    \end{minipage}
  \end{table}

\end{frame}

\begin{frame}
  \frametitle{História do GNU/Hurd}

  \begin{table}
    \begin{minipage}[t]{\linewidth}
      \color{gray}
      \rule{\linewidth}{1pt}
      \ytl{2011}{GNU Hurd 0.4}
      \ytl{2013}{Debian GNU/Hurd, GNU Hurd 0.5}
      \ytl{2015}{GNU Hurd 0.6}
      \ytl{2016}{GNU Hurd 0.8}
      \rule{\linewidth}{1pt}
    \end{minipage}
  \end{table}
  \begin{figure}[h]
    \includegraphics[scale=0.4]{imgs/xkcd_1508.png}
    \caption{\url{https://xkcd.com/1508/}}
  \end{figure}
\end{frame}

\begin{frame}[fragile]
  \frametitle{GNU HURD}

  HURD\@: Hird of Unix-Replacing Daemons

  HIRD\@: Hurd of Interfaces Representing Depth

  \begin{lstlisting}[mathescape=true,showstringspaces=false,numbers=none,frame=single]
  GNU HURD $:=$ [
    GNU  $:=$ GNU's Not Unix
    HURD $:=$ [
      HIRD $:=$ [
        HURD $:=$ [
          $\ldots$
        ] of Interfaces Representing Depth
      ] of Unix-Replacing Daemons
    ]
  ]
  \end{lstlisting}

\end{frame}

\begin{frame}
  \frametitle{Documentação}

  \begin{itemize}
    \item Horrível
    \item Escasso
    \item IRC
    \item Confuso
    \item Alinear
    \item Feito pela comunidade
    \item Página de \textit{open issues}
    \item Recompensas (\$\$\$) para quem resolver
  \end{itemize}
\end{frame}

\section{Arquitetura Geral}

\begin{frame}
  \frametitle{Arquitetura do GNU Hurd}

  \begin{itemize}
    \item Microkernel (Mach)
    \item Multiservidor
    \item GNU C
    \item Servidores flexíveis (nível usuário)
    \item Servidores compilados como quiser
    \item MIG (Mach Interface Generator)
  \end{itemize}
\end{frame}

\section{Microkernel/Mach}

\begin{frame}
  \frametitle{Kernel}
  \begin{figure}[h]
    \hspace*{-0.9cm}
    \includegraphics[scale=0.235]{imgs/kernel.png}
    \caption{\url{https://en.wikipedia.org/wiki/File:OS-structure2.svg}}
  \end{figure}
\end{frame}

\begin{frame}
  \frametitle{Comparação}
  \begin{description}
    \item[Microkernel] (e.g.\ Mach, kernel do Minix)
      \begin{itemize}
        \item Mais seguro
        \item Comunicação por mensagens (MIG)
        \item Responsabilidade bem definida
        \item Flexível (fácil de debugar os diferentes servidores)
        \item Mais estável
        \item Somente o absolutamente necessário
      \end{itemize}
    \item[Monolítico] (e.g.\ Linux)
      \begin{itemize}
        \item Não é necessário passar mensagem
        \item Mais rápido por ter acesso direto
        \item Código passa a se tornar complicado e confuso (spaghetti code)
        \item Pequenas (a às vezes aparentemente irrelevantes) mudanças podem quebrar o sistema
        \item Mais difícil de manter código
        \item Novos desenvolvedores sofrem tentando aprender código confuso
      \end{itemize}
  \end{description}
\end{frame}

\begin{frame}
  \frametitle{História do Mach}
  \begin{itemize}
    \item Criado em 1984 na Carnegie-Mellon
    \item Desenvolvimento do Mach terminou em 1994
    \item GNU Hurd usa GNU Mach (versão modificada para ser \textit{free} e manter-se atualizado)
    \item Microkernel/Nanokernel (alguns consideram ``Hybrid Kernel'')
    \item Mensagens por meio do MIG (Mach Interface Generator)
    \item MIG possibilita rodar código por RPC
  \end{itemize}
\end{frame}

\begin{frame}
  \frametitle{O que Mach faz e não faz?}
  \begin{description}
    \item[Faz:]~\\
      \begin{itemize}
        \item Gerenciamento de Memória
        \item Gerenciamento de Processos
        \item Comunicações (mensagens dos outros servidores)
        \item Entrada e Saída
      \end{itemize}
    \item [Não faz:]~\\
      \begin{itemize}
        \item Sistema de Arquivos
        \item Drivers
        \item Aplicativos de Usuário (WM, DE, etc.)
      \end{itemize}
  \end{description}
\end{frame}

\section{Multiservidor}

\begin{frame}
  \frametitle{Multiservidor}
  \begin{itemize}
    \item Rodam paralelamente ao Mach
    \item Comunicam-se pelo MIG
    \item Ficam na camada de usuário
    \item Qualquer linguagem
    \item Compilado como quiser
    \item Debugar enquanto os outros servidores rodam
    \item Independente de todos os outros servidores e kernel
    \item Completamente modificável
    \item Seguro (ficam em camada de usuário)
    \item Não é preciso dar reboot para testar
  \end{itemize}
\end{frame}

\begin{frame}
  \frametitle{Alguns exemplos de servidores}
  \begin{description}
    \item[Core]~\\
      \begin{itemize}
        \item \code{auth} privilégios, senhas e identificação
        \item \code{crash} erros
        \item \code{exec} rodar executáveis
        \item \code{fifo} pipes
        \item \code{firmlink} ``half-way between a symbolic link and hard link''
        \item \code{ifsock} sockets
        \item \code{init} boot
        \item \code{null} equivalente a \code{/dev/null} e \code{/dev/zero}
        \item \code{procs} PIDs
        \item \code{term} terminal
        \item $\ldots$
      \end{itemize}
    \item[Filesystem]~\\
      \begin{itemize}
        \item \code{ext2fs}, \code{isofs}, \code{nfs}, \code{ufs}, \code{ftpfs}, \code{storeio}
      \end{itemize}
  \end{description}
\end{frame}

\section{Memória}

\begin{frame}
  \frametitle{Memória no GNU Mach}
  \begin{itemize}
    \item 32-bit
    \item 2GB kernelspace
    \item 2GB userspace
    \item Mach cuida de toda memória virtual
    \item Pedidos de memória pelos processos feitos por mensagens
    \item Mach retorna um ponteiro para a memória e o tamanho alocado
  \end{itemize}
\end{frame}

\begin{frame}
  \frametitle{Escalonamento}
  Gerenciamento de memória por meio de uma Red-Black Tree (RBT) e usando Best-fit.

  Red-black tree é uma BST com espaço $\bigo(n)$ e operações $\bigo(\log n)$.

  Blocos: lista duplamente ligada.

  Usa-se uma RBT para ordenar os blocos em tamanho. Acha o melhor fit em tempo $\bigo(\log n)$.
  Modificar a lista é feito em $\bigo(1)$ já que é duplamente ligada.

  Complexidade final é $\bigo(\log n)$.

  (BSD, Linux, entre outros)
\end{frame}

\section{Escalonamento de Processos}

\begin{frame}
  \frametitle{Processos no GNU Mach}
  \begin{description}
    \item[Kernelspace:]~\\
      \begin{itemize}
        \item Usa \textit{continuations} (estrutura que guarda informação do processo)
        \item Continuations usam menos espaço pois não precisam da pilha de threads do kernel
        \item Não é preemptivo (por causa de continuations)
        \item Ser preemptivo requer toda a pilha de threads do kernel
        \item Escalonamento feito pelo Mach
        \item Escalonamento 1:1 (igual a Solaris, NetBSD, FreeBSD, OS X, iOS, OS/2 e Win32)
        \item Escalonamento feito por filas de prioridade
      \end{itemize}
  \end{description}
\end{frame}

\begin{frame}
  \frametitle{Processos no GNU Mach}
  \begin{description}
    \item[Userspace:]~\\
      \begin{itemize}
        \item Threads
        \item \code{libthreads}
        \item \code{libports}
        \item Servidores tem maior prioridade que user threads
        \item Não suporta múltiplas CPUs (anos 90)
      \end{itemize}
  \end{description}
\end{frame}

\section{Sistema de Arquivos}

\begin{frame}
  \frametitle{File system no GNU Hurd}
  \begin{itemize}
    \item Por \textit{i-nodes}
    \item \code{ext2fs}
  \end{itemize}
\end{frame}

\section{Comparação}

%--------------------------------------------------------------------------------------------------

\section[Referências]{Referências e Bibliografia}
\begin{frame}[t,allowframebreaks]
  \frametitle{Referências e Bibliografia}
  \footnotesize
  \nocite{*}
  \printbibliography[]
\end{frame}

\end{document}
