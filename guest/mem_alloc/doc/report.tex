\documentclass{amsart}

\usepackage[brazilian]{babel}
\usepackage[utf8]{inputenc}
\usepackage{graphicx}
\usepackage{mathtools}
\usepackage{amsthm}
\usepackage{amsfonts}
\usepackage{hyperref}
\usepackage[singlelinecheck=false]{caption}
\usepackage{enumitem}
\usepackage[justification=centering]{caption}
\usepackage{indentfirst}
\usepackage{listings}
\usepackage{minted}

\makeatletter
\def\subsection{\@startsection{subsection}{3}%
  \z@{.5\linespacing\@plus.7\linespacing}{.1\linespacing}%
  {\normalfont}}
\makeatother

\DeclareMathOperator*{\argmin}{arg\,min}
\DeclareMathOperator*{\argmax}{arg\,max}

\newcommand\defeq{\mathrel{\overset{\makebox[0pt]{\mbox{\normalfont\tiny\sffamily def}}}{=}}}

\captionsetup[table]{labelsep=space}

\theoremstyle{plain}

\newtheorem*{definition}{Definição}
\newtheorem{theorem}{Theorem}
\newtheorem{proposition}{Proposition}
\newtheorem{exercise}{Exercise}

\newcommand{\set}[1]{\mathcal{#1}}
\newcommand{\pr}{\mathbb{P}}
\renewcommand{\implies}{\Rightarrow}
\newcommand{\prompt}{$\$$}

\newcommand{\code}[1]{\lstinline[mathescape=true]{#1}}
\newcommand{\mcode}[1]{\lstinline[mathescape]!#1!}

\lstset{%
  language=C,
  numbers=left,
  breaklines=true,
  keywordstyle=\bfseries,
  basicstyle=\ttfamily
}

\lstdefinestyle{numbers}{numbers=left, stepnumber=1, numberstyle=\tiny, numbersep=10pt}
\lstdefinestyle{nonumbers}{numbers=none}

\setlength{\parskip}{1em}

\title[]{\rule{10.5cm}{0.8pt}\\Exercício-Programa 2:\\
Escalonamento de Processos
\\\vspace{2mm}\footnotesize
  Sistemas Operacionais --- MAC0422\\\rule{10cm}{0.8pt}}
\author[]{Renato Lui Geh\\NUSP\@: 8536030\\
          Guilherme Freire\\NUSP\@: 7557373}

\begin{document}
\date{\today}
\maketitle

\section{Introdução}

O EP foi feito em um Minix 3.1.2a simulado pela VM VirtualBox. Os arquivos fonte estão localizados
em \code{/usr/local/}.

Os arquivos modificados foram:

\begin{itemize}
  \item \code{/usr/local/include/unistd.h}
  \item \code{/usr/local/include/callnr.h}
  \item \code{/usr/local/src/include/unistd.h}
  \item \code{/usr/local/src/include/callnr.h}
  \item \code{/usr/local/src/lib/posix/Makefile.in}
  \item \code{/usr/local/src/servers/fs/misc.c}
  \item \code{/usr/local/src/servers/fs/proto.h}
  \item \code{/usr/local/src/servers/fs/table.c}
  \item \code{/usr/local/src/servers/pm/alloc.c}
  \item \code{/usr/local/src/servers/pm/proto.h}
  \item \code{/usr/local/src/servers/pm/table.c}
\end{itemize}

As versões modificadas estão em \code{/usr/local/}, assim como os arquivos não modificados. Deste
jeito, pode-se rodar \code{/usr/local/src/tools/Makefile} sem alterar o código original. Três
arquivos foram adicionados:

\begin{itemize}
  \item \code{/usr/local/src/lib/posix/_alloc_algorithm.c}
\end{itemize}

Quando os blocos de código transcritos neste relatório não forem muito grandes, vamos indicar as
modificações feitas. Um símbolo \code{-} no início da linha indica a linha original no Minix. Um
símbolo \code{+} no início da linha indica a nova linha adaptada para o EP\@. Uma linha vazia com o
símbolo \code{-} indica que no código original a linha não existia. Analogamente, \code{+} em uma
linha vazia indica que deletamos a linha original. Um \code{#} indica um comentário no código, ou
seja, a linha indicada por este símbolo não existe no arquivo original.

\section{Alterando a política}

Foram adicionadas as macros

\begin{lstlisting}[frame=leftline,mathescape=true,style=nonumbers]
-
+ #define FIRST_FIT  0
+ #define WORST_FIT  1
+ #define BEST_FIT   2
+ #define RANDOM_FIT 3
\end{lstlisting}
e o protótipo de função.
\begin{lstlisting}[frame=leftline,mathescape=true,style=nonumbers]
-
+ _PROTOTYPE(void alloc_algorithm, (int _policy));
\end{lstlisting}

Adiciona-se a macro que define a chamada de sistema:

\begin{lstlisting}[frame=leftline,mathescape=true,style=nonumbers]
-
+ #define ALLOC_ALGORITHM 58
\end{lstlisting}

No arquivo \code{lib/posix/_alloc_algorithm.c} implementamos a função que passa a mensagem para os
servidores:

\begin{minted}[linenos=true,gobble=2,frame=lines]{c}
  #include <lib.h>
  #define alloc_algorithm _alloc_algorithm
  #include <unistd.h>
  #include <stdio.h>

  PUBLIC void alloc_algorithm(_policy)
  int _policyç
  {
    message m;
    m.m1_i1 = _policy;
    return _syscall(MM, ALLOC_ALGORITHM, &m);
  }
\end{minted}

Note como o argumento \code{_policy} é mandado como mensagem para \code{_syscall}. Usaremos esta
mensagem quando formos implementar a função nos servidores. Em seguida, adicionamos a nova função à
tabela dos servidores (\code{pm} e \code{fs}).

\begin{lstlisting}[frame=leftline,mathescape=true,style=nonumbers]
- no_sys, /* 58 = unused */
+ do_alloc_algorithm, /* 58 = ALLOC_ALGORITHM */
\end{lstlisting}

E adicionamos o protótipo da função em \code{proto.h}.

\begin{lstlisting}[frame=leftline,mathescape=true,style=nonumbers]
-
+ _PROTOTYPE(int do_alloc_algorithm, (int policy));
\end{lstlisting}

E em seguida implementamos a nova chamada de sistema em \code{pm/alloc.c}.

\begin{minted}[linenos=true,gobble=2,frame=lines]{c}
  PUBLIC int do_alloc_algorithm(policy)
  int policy;
  {
    /* Gets argument policy from message. */
    policy = m_in.m1_i1;
    if (policy != FIRST_FIT && policy != BEST_FIT
          && policy != WORST_FIT && policy != RANDOM_FIT)
        return EINVAL; /* invalid argument error as defined in errno.h */
    alloc_policy = policy;
    return OK;
  }
\end{minted}

Linha 5 refere-se a mensagem que mandamos por \code{lib/posix/_alloc_algorithm.c}. Verificamos se
a política enviada não é uma política válida. Se tal erro ocorre, retornamos \code{EINVAL}, que é o
sinal de argumento inválido. Senão, atualizamos uma variável global \code{alloc_policy} com o novo
valor.

A função de usuário para mudar a política é dado por \code{change_allocation_policy.c} em
\code{/root}.

\begin{minted}[linenos=true,gobble=2,frame=lines]{c}
  #include <stdio.h>
  #include <unistd.h>
  #include <string.h>

  int main(int argc, char *args[]) {
    int pol;
    char *str;

    if (argc != 2) {
      printf(``Usage:\n  %s policy\nArguments:\n''
          ``  policy - Indicates which policy to use:\n''
          ``    first_fit, worst_fit, best_fit, random_fit\n'', args[0]);
      return 1;
    }

    str = args[1];
    if (!strcmp(str, ``first_fit'')) {
      pol = FIRST_FIT;
    } else if (!strcmp(str, ``worst_fit'')) {
      pol = WORST_FIT;
    } else if (!strcmp(str, ``best_fit'')) {
      pol = BEST_FIT;
    } else if (!strcmp(str, ``random_fit'')) {
      pol = RANDOM_FIT;
    } else {
      puts(``Argument invalid.'');
      return 2;
    }

    alloc_algorithm(pol);

    return 0;
  }
\end{minted}

Esta função apenas faz a comparação de strings para descobrir qual política o usuário deseja
atualizar o sistema com. Em seguida, chama a função \code{alloc_algorithm} que acabamos de
descrever.

\section{Políticas de alocação}

Nesta seção descreveremos como foram feitas as diferentes políticas. O Minix já implementa
\code{FIRST_FIT} em \code{servers/pm/alloc.c} na função \code{alloc_mem}. Esta implementação serviu
de base para as outras implementações.

\subsection{Worst fit}

\begin{minted}[linenos=true,gobble=2,frame=lines]{c}
  prev_ptr = hole_head;
  hp = hole_head->h_next;

  cand_prev_ptr = NIL_HOLE;
  candidate = hole_head;
  candidate_flag = 0;

  if (candidate->h_len >= clicks)
    candidate_flag = 1;

  while (hp != NIL_HOLE && hp->h_base < swap_base) {
    if (hp->h_len > candidate->h_len && hp->h_len >= clicks) {
      cand_prev_ptr = prev_ptr;
      candidate = hp;
      candidate_flag = 1;
    }
    prev_ptr = hp;
    hp = hp->h_next;
  }

  if (candidate_flag) {
    old_base = candidate->h_base;
    candidate->h_base += clicks;
    candidate->h_len -= clicks;

    if(candidate->h_base > high_watermark)
      high_watermark = candidate->h_base;

    if (candidate->h_len == 0)
      del_slot(cand_prev_ptr, candidate);

    return(old_base);
  }
\end{minted}

Nesta política, escolhemos o buraco de memória que seja maior para adicionarmos o novo processo. Na
linha 4-6 definimos o ponteiro para o buraco anterior ao candidato, o ponteiro para o candidato e
uma \textit{flag} para definir se achamos um possível candidato. As linhas 8-9 apenas verificam se
a própria cabeça da lista é um possível candidato. Em seguida, iteramos pela lista. Se o tamanho do
buraco for o suficiente para a memória do processo (\code{clicks}) e o tamanho de tal buraco excede
o do nosso atual candidato, então atualizamos o candidato. Linha 21 ocorre quando já escolhemos um
candidato. Se tal candidato for exatamente o tamanho da memória necessária (linha 29), então
deletamos o buraco. Atualizamos o buraco candidato e retornamos o local de memória a ser usado.
Caso a linha 21 retorne falso, então tentaremos fazer o \code{swap} da memória. Se mesmo assim não
for possível, então retornaremos \code{NO_MEM}, que indica que não há memória suficiente.

\subsection{Best fit}

\begin{minted}[linenos=true,gobble=2,frame=lines]{c}
  prev_ptr = hole_head;
  hp = hole_head->h_next;

  cand_prev_ptr = NIL_HOLE;
  candidate = hole_head;
  candidate_flag = 0;

  if (candidate->h_len >= clicks)
    candidate_flag = 1;

  while (hp != NIL_HOLE && hp->h_base < swap_base) {
    if (!candidate_flag && hp->h_len >= clicks) {
      cand_prev_ptr = prev_ptr;
      candidate = hp;
      candidate_flag = 1;
    }
    if (hp->h_len < candidate->h_len && hp->h_len >= clicks) {
      cand_prev_ptr = prev_ptr;
      candidate = hp;
      candidate_flag = 1;
    }
    prev_ptr = hp;
    hp = hp->h_next;
  }

  if (candidate_flag) {
    old_base = candidate->h_base;
    candidate->h_base += clicks;
    candidate->h_len -= clicks;

    if(candidate->h_base > high_watermark)
      high_watermark = candidate->h_base;

    if (candidate->h_len == 0)
      del_slot(cand_prev_ptr, candidate);

    return(old_base);
  }
\end{minted}

Na política \textit{best fit}, queremos achar o buraco que tenha o menor possível tamanho e ainda
seja suficiente para manter a memória do processo. Assim como em \textit{worst fit}, temos um
candidato que indica qual o buraco a ser modificado (ou potencialmente removido). A única diferença
entre worst fit e best fit é a condição na qual escolhemos o candidato. Note que a linha 17 escolhe
um buraco que tenha o menor possível tamanho mas que ainda seja maior ou igual a \code{clicks}.
Assim como em worst fit, após acharmos o melhor candidato, atualizamos o buraco (potencialmente
removendo se o tamanho foi exatamente igual ao do requisitado) e em seguida retornamos o local da
memória.

\subsection{Random fit}

\begin{minted}[linenos=true,gobble=2,frame=lines]{c}
  hp = hole_head;

  possible_candidates = 0;

  while (hp != NIL_HOLE && hp->h_base < swap_base) {
    if (hp->h_len >= clicks) {
      possible_candidates++;
    }
    hp = hp->h_next;
  }

  prev_ptr = NIL_HOLE;
  hp = hole_head;

  if (possible_candidates > 0) {
    selected = (random() \% possible_candidates) + 1;
    i = 0;
    while (hp != NIL_HOLE && hp->h_base < swap_base) {
      if (hp->h_len >= clicks) {
        i++;
        if (i == selected) {
          old_base = hp->h_base;
          hp->h_base += clicks;
          hp->h_len -= clicks;

          if(hp->h_base > high_watermark)
            high_watermark = hp->h_base;

          if (hp->h_len == 0) del_slot(prev_ptr, hp);

          return(old_base);
        }
      }
      prev_ptr = hp;
      hp = hp->h_next;
    }
  }
\end{minted}

Em \textit{random fit}, queremos um buraco aleatória que possa conter a memória do processo. Para
isso, contamos o número de candidatos possíveis (\code{possible_candidates}) e em seguida
escolhemos algum que esteja neste intervalo. Caso não hajam candidatos possíveis, retornamos sem
memória.

\section{\code{memstat}}

O arquivo \code{memstat.c} em \code{/root} imprime a média, mediana e desvio padrão dos buracos a
cada segundo.

\begin{minted}[linenos=true,gobble=2,frame=lines]{c}
  /*getsysinfo(MM, SI_MEM_ALLOC, &store);*/
  void getsysinfo(int who, int what, void *where) {
    message m;
    m.m1_i1 = what;
    m.m1_p1 = where;
    _syscall(who, GETSYSINFO, &m);
  }
\end{minted}

Para recuperar o estado da lista de buracos, precisamos chamar uma chamada de sistema chamada
\code{getsysinfo}. Como esta chamada de sistema não tem função de usuário equivalente, temos de
chama-la por meio de uma \code{syscall}. Para tanto, enviamos as duas mensagens \code{what} e
\code{where}, que equivalem a o quê desejamos procurar e onde guardar a mensagem. No caso de
\code{memstat}, desejamos descobrir a situação de alocação de memória (\code{SI_MEM_ALLOC}) e
iremos guardar tal informação em um \code{struct pm_mem_info}, que representa uma lista de buracos.

\begin{minted}[linenos=true,gobble=2,frame=lines]{c}
  int mem_data(struct hole *holes, double *mean, double *median, double *stddev) {
    int n, i, t;
    struct hole *it = holes;
    *mean = *median = *stddev = 0;
    for (n = 0; it != NULL; ++n) {
      *mean += (double) it->h_len;
      it = it->h_next;
    }
    if (n == 0) {
      *median = *stddev = 0;
      return n;
    }
    *mean /= (double) n;
    t = n/2;
    it = holes;
    for (i = 0; i < n; i++) {
      double k = (double) it->h_len - *mean;
      if (i == t)
        *median = (double) it->h_len;
      *stddev += k*k;
      it = it->h_next;
    }
    *stddev = sqrt(*stddev/n);
    return n;
  }
\end{minted}

A função \code{mem_data} itera pela lista de buracos \code{struct hole *holes} e computa a média,
mediana e desvio padrão. Ao final, retorna o número de elementos e guarda os valores nos endereços
dados.

\section{Testes}

\section{Observações importantes quanto a execução do EP}

Como usamos um Floppy Controller na nossa VM, a tela inicial irá indicar que não foi encontrado um
local de boot. Para resolver isto, pressione \code{F12} e em seguida pressione \code{1}. Isto
selecionará o controlador principal (a que possue a imagem do Minix) como local de boot.

Como não mudamos o caminho de boot padrão do Minix, quando a VM for rodada, deve-se dar boot na
imagem correta. Por padrão, a VM irá dar boot na imagem padrão original.

É recomendável que se recompile o Minix novamente para garantir que tudo esteja o mais recente
possível. Caso não se recompile o Minix, a imagem em \code{/boot/image} mais recente é:

\begin{lstlisting}[frame=leftline,mathescape=true,style=nonumbers]
/boot/image/3.1.2ar44
\end{lstlisting}

Para rodar a imagem escolhida, basta indicar o caminho. Por exemplo, caso a imagem desejada seja
\code{/boot/image/3.1.2ar44}, então:

\begin{lstlisting}[frame=leftline,mathescape=true,style=nonumbers]
# Garanta que esteja na tela de boot.
shutdown
# Indique qual imagem deve ser escolhida.
image=/boot/image/3.1.2ar44
# Faca o boot.
boot
\end{lstlisting}

\end{document}
