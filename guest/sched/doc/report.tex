\documentclass{amsart}

\usepackage[brazilian]{babel}
\usepackage[utf8]{inputenc}
\usepackage{graphicx}
\usepackage{mathtools}
\usepackage{amsthm}
\usepackage{amsfonts}
\usepackage{hyperref}
\usepackage[singlelinecheck=false]{caption}
\usepackage{enumitem}
\usepackage[justification=centering]{caption}
\usepackage{indentfirst}
\usepackage{listings}
\usepackage{minted}

\makeatletter
\def\subsection{\@startsection{subsection}{3}%
  \z@{.5\linespacing\@plus.7\linespacing}{.1\linespacing}%
  {\normalfont}}
\makeatother

\DeclareMathOperator*{\argmin}{arg\,min}
\DeclareMathOperator*{\argmax}{arg\,max}

\newcommand\defeq{\mathrel{\overset{\makebox[0pt]{\mbox{\normalfont\tiny\sffamily def}}}{=}}}

\captionsetup[table]{labelsep=space}

\theoremstyle{plain}

\newtheorem*{definition}{Definição}
\newtheorem{theorem}{Theorem}
\newtheorem{proposition}{Proposition}
\newtheorem{exercise}{Exercise}

\newcommand{\set}[1]{\mathcal{#1}}
\newcommand{\pr}{\mathbb{P}}
\renewcommand{\implies}{\Rightarrow}
\newcommand{\prompt}{$\$$}

\newcommand{\code}[1]{\lstinline[mathescape=true]{#1}}
\newcommand{\mcode}[1]{\lstinline[mathescape]!#1!}

\lstset{%
  language=C,
  numbers=left,
  breaklines=true,
  keywordstyle=\bfseries,
  basicstyle=\ttfamily
}

\lstdefinestyle{numbers}{numbers=left, stepnumber=1, numberstyle=\tiny, numbersep=10pt}
\lstdefinestyle{nonumbers}{numbers=none}

\setlength{\parskip}{1em}

\title[]{\rule{10.5cm}{0.8pt}\\Exercício-Programa 2:\\
Escalonamento de Processos
\\\vspace{2mm}\footnotesize
  Sistemas Operacionais --- MAC0422\\\rule{10cm}{0.8pt}}
\author[]{Renato Lui Geh\\NUSP\@: 8536030\\
          Guilherme Freire\\NUSP\@: }

\begin{document}
\date{\today}
\maketitle

\section{Introdução}

O EP foi feito em um Minix 3.1.2a simulado pela VM VirtualBox. Os arquivos fonte estão localizados
em \code{/usr/local/src/}.

Os arquivos modificados foram:

\begin{itemize}
  \item \code{/usr/local/src/kernel/proc.c}
  \item \code{/usr/local/src/kernel/proc.h}
\end{itemize}

Quando os blocos de código transcritos neste relatório não forem muito grandes, vamos indicar as
modificações feitas. Um símbolo \code{-} no início da linha indica a linha original no Minix. Um
símbolo \code{+} no início da linha indica a nova linha adaptada para o EP\@. Uma linha vazia com o
símbolo \code{-} indica que no código original a linha não existia. Analogamente, \code{+} em uma
linha vazia indica que deletamos a linha original.

\section{Kernel}

As modificações no kernel foram feitas nos arquivos:

\begin{enumerate}
  \item\label{proc_c} \code{/usr/local/src/kernel/proc.c}
  \item\label{proc_h} \code{/usr/local/src/kernel/proc.h}
\end{enumerate}

No arquivo~\ref{proc_h}, adicionamos uma nova macro \code{BATCH_Q}:

\begin{lstlisting}[frame=leftline,mathescape=true,style=nonumbers]
-
+ #define BATCH_Q 15 /* batch queue, before IDLE and after user queues. */
\end{lstlisting}

Esta nova macro indica uma nova fila de prioridade, antes da fila \code{idle} (\code{IDLE_Q}) e
depois da última fila de prioridade de usuário (\code{MIN_USER_Q}). Como adicionamos uma nova fila,
precisamos incrementar em um a macro que indica o número total de filas:

\begin{lstlisting}[frame=leftline,mathescape=true,style=nonumbers]
- #define NR_SCHED_QUEUES 16
+ #define NR_SCHED_QUEUES 17
\end{lstlisting}

Além disso, como estamos transladando a fila de \code{idle}, precisamos incrementa-la também.

\begin{lstlisting}[frame=leftline,mathescape=true,style=nonumbers]
- #define IDLE_Q 15
+ #define IDLE_Q 16
\end{lstlisting}

Estas modificações em \code{proc.h} concluem a tarefa 1 do EP\@. Agora discutiremos as mudanças
feitas em \code{proc.c}, que coincidem justamente com a tarefa 3.

Seguindo a convenção ANSI descrita em \code{/usr/lib/ansi.h} e seguida pelo Minix 3.1.2a, devemos
primeiro declarar a nova função de escalonamento em batch:

\begin{lstlisting}[frame=leftline,mathescape=true,style=nonumbers]
-
+ FORWARD _PROTOTYPE (void sched_batch, (struct proc *rp, int *queue, int *front));
\end{lstlisting}

Vamos agora transcrever e analisar a nova função \code{sched_batch} que trata do escalonamento em
batch.

\begin{minted}[linenos=true,gobble=2,frame=lines]{c}
  /*=================================================================*
   *        sched_batch          *
   *=================================================================*/

  PRIVATE void sched_batch(rp, queue, front)
  register struct proc *rp;  /* process to be scheduled */
  int *queue;   /* return: queue to use */
  int *front;   /* return: front or back */
  {
    register struct proc *batch_it;
    int lmin, diff;

    batch_it = rdy_head[BATCH_Q];
    diff = 0;
    lmin = -1;
    if (batch_it->p_time_left <= 0) {
      /* Find `last' proc wrt user time */
      for (; batch_it != NIL_PROC; batch_it = batch_it->p_nextready) {
        if (batch_it == rp) continue;
        if (lmin < 0)
          lmin = batch_it->p_user_time;
        else if (batch_it->p_user_time < lmin)
          lmin = batch_it->p_user_time;
      }
      /* Invariant: diff >= 0, since lmin is minimum */
      diff = rp->p_user_time - lmin;
      /* If diff == 0, rp is next to lmin => rp must go front */
      if (diff <= 0)
        rp->p_ticks_left = rp->p_quantum_time;
      /* Else, rp is `in front` of lmin => rp must wait for last proc */
      else
        rp->p_ticks_left = 0;
    }
    *queue = BATCH_Q;
    /* If there is still time left, keep it front. Else, depends on diff. */
    *front = !diff;
  }
\end{minted}

A função \code{sched_batch}, assim como a função original \code{sched} do Minix, toma como
argumentos uma \code{struct proc*} que representa o endereço do processo a ser escalonado, e dois
endereços para inteiros, \code{queue} que sinaliza qual a fila de prioridade para se usar, e
\code{front} que indica se o processo deve ir na frente da fila ou atrás.

Vamos analisar a função. Antes de mais nada, declaramos as variáveis que iremos utilizar. A
variável \code{register struct proc *batch_it} será, no \code{for} da linha 18, o endereço para
processo de cada item da fila de prioridade indexado por \code{BATCH_Q}. O inteiro \code{lmin}
é o menor tempo de usuário de todos os processos da fila. Já \code{diff} será usado para medir a
diferença entre o menor tempo de usuário e o tempo de usuário do processo a ser escalonado.

As linhas 18-24 apenas acham o menor tempo de execução da fila de processos em batch. Percorremos
a fila e consideramos os tempos de execução de usuário desde que o processo visto é distinto
daquele que estamos escalonando. Em seguida, na linha 26, atribuímos o valor da diferença entre o
menor tempo ao tempo de execução de usuário do processo a ser escalonado. Note a invariância de que
nesta linha \mcode{diff $\geq 0$}. Isso ocorre pois, como \code{lmin} é mínimo, então se
considerarmos o oposto, então \code{rp->p_user_time < lmin}, o que é uma contradição.

Note que na linha 14, inicializamos a variável \code{diff} como 0. Isso ocorre pois devemos
considerar dois casos. No caso em que o processo ainda possue tempo de execução (ou seja, se a
linha 16 retornar falso) devemos colocar o processo no começo da fila. Como diff é inicializado
como 0, a linha 36 funciona como intencionado. Agora considere o caso em que o processo precisa ser
re-escalonado (ou seja, se a linha 16 retornar verdadeiro). Neste caso, vamos ter dois subcasos:

\begin{enumerate}[label*=\arabic*o.~subcaso]
  \item~\\
    \begin{enumerate}[label=1.\arabic*.]
      \item Temos que \code{diff == 0} (linha 28).
      \item Ou seja, \code{rp} está tão ``atrasado'' quanto o processo que rodou menos.
      \item Isto indica que devemos rodar \code{rp} antes, já que ele está empatado com o último.
      \item Portanto, \code{rp} deve ir ``na frente'' da fila. Como \code{diff} é 0, \code{*front=1}.
      \item Além disso, damos um tempo de ticks igual ao número de ticks equivalente a um quantum.
    \end{enumerate}
  \item~\\
  \begin{enumerate}[label=2.\arabic*.]
    \item No segundo caso, \code{diff > 0} (linha 31).
    \item Ou seja, \code{rp} está ``adiantado'' em relação ao processo mais ``atrasado''.
    \item Isto indica que devemos rodar o processo mais atrasado antes de \code{rp}.
    \item Portanto, \code{rp} deve ir no final da fila e dar passagem para os processos atrasados,
      e já que \mcode{diff $> 0$}, então \code{*front=0}.
    \item Como vamos ``pular'' este processo, damos um tempo de 0 ticks restantes e pomos no final
      da fila.
  \end{enumerate}
\end{enumerate}

Perceba que, quando todos os processos ``empatarem'' em relação ao tempo de execução, um deles será
escolhido e dado um tempo equivalente a um quantum de tempo. Quando este terminar, todos os outros
processos também percorrerão um quantum de tempo cada um, um de cada vez. Além disso, todos os
processos que estão na frente esperarão os atrasados. Isso é a definição de escalonamento
\textit{Round Robin}, como foi pedido no enunciado quando todos os processos empatam.

Ao final da função, anunciamos que a fila de prioridade a ser usada é aquela indexada por
\code{BATCH_Q} (linha 34). Em seguida, atribuímos o valor de \code{diff} como descrito
anteriormente.

Agora resta chamarmos a função de escalonamento. Faremos isso dentro da função \code{enqueue}, no
próprio \code{proc.c}.

\begin{minted}[linenos=true,gobble=2,frame=lines,mathescape=true]{c}
  PRIVATE void enqueue(rp)
  register struct proc *rp; /* this process is now runnable */
  {
    /* $\ldots$ */
    /* Determine where to insert to process. */
      /* ########################################################## */
      if (rp->p_priority == BATCH_Q)
        sched_batch(rp, &q, &front);
      else
      /* ########################################################## */
        sched(rp, &q, &front);
    /* $\ldots$ */
  }
\end{minted}

Quando o processo a ser escalonado tem prioridade igual a \code{BATCH_Q}, escalonamos com a função
\code{sched_batch}, senão escalonamos normalmente com \code{sched}.

\end{document}
