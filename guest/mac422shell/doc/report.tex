\documentclass{amsart}

\usepackage[brazilian]{babel}
\usepackage[utf8]{inputenc}
\usepackage{graphicx}
\usepackage{mathtools}
\usepackage{amsthm}
\usepackage{amsfonts}
\usepackage{hyperref}
\usepackage[singlelinecheck=false]{caption}
\usepackage{enumitem}
\usepackage[justification=centering]{caption}
\usepackage{indentfirst}
\usepackage{listings}
\usepackage{minted}

\makeatletter
\def\subsection{\@startsection{subsection}{3}%
  \z@{.5\linespacing\@plus.7\linespacing}{.1\linespacing}%
  {\normalfont\itshape}}
\makeatother

\DeclareMathOperator*{\argmin}{arg\,min}
\DeclareMathOperator*{\argmax}{arg\,max}

\newcommand\defeq{\mathrel{\overset{\makebox[0pt]{\mbox{\normalfont\tiny\sffamily def}}}{=}}}

\captionsetup[table]{labelsep=space}

\theoremstyle{plain}

\newtheorem*{definition}{Definição}
\newtheorem{theorem}{Theorem}
\newtheorem{proposition}{Proposition}
\newtheorem{exercise}{Exercise}

\newcommand{\set}[1]{\mathcal{#1}}
\newcommand{\pr}{\mathbb{P}}
\renewcommand{\implies}{\Rightarrow}
\newcommand{\prompt}{$\$$}

\newcommand{\code}[1]{\lstinline[mathescape=true]{#1}}
\newcommand{\mcode}[1]{\lstinline[mathescape]!#1!}

\lstset{%
  language=C,
  numbers=left,
  breaklines=true,
  keywordstyle=\bfseries,
  basicstyle=\ttfamily
}

\lstdefinestyle{numbers}{numbers=left, stepnumber=1, numberstyle=\tiny, numbersep=10pt}
\lstdefinestyle{nonumbers}{numbers=none}

\setlength{\parskip}{1em}

\title[]{\rule{10.5cm}{0.8pt}\\Exercício-Programa 1: mac422shell\\\vspace{2mm}\footnotesize
  Sistemas Operacionais -- MAC0422\\\rule{10cm}{0.8pt}}
\author[]{Renato Lui Geh\\NUSP\@: 8536030}

\begin{document}
\date{\today}
\maketitle

\section{Introdução}

O EP foi feito em um Minix 3.1.2a simulado pela VM VirtualBox. Os arquivos fonte estão localizados
em \code{/usr/local/src/mac422shell} e o binário compilado em \code{/usr/local/bin/} com o nome de
\code{mac422shell}.

Ao rodar o shell \code{mac422shell} o prompt do shell terá o seguinte aspecto:

\begin{lstlisting}[mathescape=true,style=nonumbers]
  [r] pwd $\$$>
\end{lstlisting}

Onde \code{r} é o valor de saída do último comando executado e \code{pwd} é o caminho absoluto em
que \code{mac422shell} foi executado.

Existem nove comandos pré-existentes na shell:

\begin{itemize}
  \item ls
    \begin{itemize}
      \item Lista os conteúdos do diretório \code{pwd}. Similar (mas não igual) a \code{rodeveja
        \\bin\\ls}.
      \item Uso: \mcode{ls}
    \end{itemize}
  \item pwd
    \begin{itemize}
      \item Retorna o diretório atual. Similar (mas não igual) a \code{rodeveja \\bin\\pwd}.
      \item Uso: \mcode{pwd}
    \end{itemize}
  \item exit
    \begin{itemize}
      \item Sai do shell.
      \item Uso: \mcode{exit}
    \end{itemize}
  \item true
    \begin{itemize}
      \item Retorna um valor de saída 0.
      \item Uso: \mcode{true}
    \end{itemize}
  \item false
    \begin{itemize}
      \item Retorna um valor de saída 1.
      \item Uso: \mcode{false}
    \end{itemize}
  \item protegepracaramba
    \begin{itemize}
      \item Dá ao arquivo dado como parâmetro proteção \code{000}.
      \item Uso: \mcode{protegepracaramba filename}
    \end{itemize}
  \item liberageral
    \begin{itemize}
      \item Dá ao arquivo dado como parâmetro proteção \code{777}.
      \item Uso: \mcode{liberageral filename}
    \end{itemize}
  \item rodeveja
    \begin{itemize}
      \item Roda um executável e em seguida retorna o valor de saída na saída padrão.
      \item Uso: \mcode{rodeveja command [arguments]}
    \end{itemize}
  \item rode
    \begin{itemize}
      \item Roda um executável em background e imprime a saída na saída padrão.
      \item Uso: \mcode{rode command [arguments]}
    \end{itemize}
\end{itemize}

Além destes comandos, podemos também rodar um comando externo dando o nome do executável. Por
exemplo:

\begin{lstlisting}[mathescape=true,style=nonumbers]
  $\$$ /bin/ls -la
\end{lstlisting}

Rodar o comando externo tem um comportamento semelhante a usar \code{rodeveja}, com a diferença de
que rodar apenas o comando não irá imprimir o valor de retorno do comando na saída padrão%
\footnote{Note que usar \code{rodeveja} ou rodar o comando diretamente irá, em ambos os casos,
modificar o valor de \code{r} no prompt.}.

\section{Organização}

Iremos agora enumerar os arquivos fontes presentes em \code{/usr/local/src} e explicar como foi
organizado o código.

O código foi separado nos seguintes arquivos:

\begin{itemize}
  \item \code{args.\{c,h\}}
  \item \code{builtin.\{c,h\}}
  \item \code{proc.\{c,h\}}
  \item \code{prompt.\{c,h\}}
  \item \code{ustring.\{c,h\}}
  \item \code{utils.\{c,h\}}
  \item \code{main.c}
\end{itemize}

O código em \code{ustring} refere-se ao tipo \code{string_t}, uma estrutura para facilitar a
passagem de argumentos. Guarda uma cadeia de caracteres e um inteiro indicando o tamanho. Em
\code{utils} encontram-se algumas funções de uso geral, como macros para imprimir mensagens de
erro, achar o mínimo ou máximo entre dois inteiros ou retornar o diretório atual com a chamada de
sistema \code{getcwd}. Os arquivos \code{args} referem-se a estrutura que guarda argumentos
dinamicamente. As funções presentes nestes arquivos manipulam a estrutura \code{args_t}, como
veremos mais adiante.

Os arquivos fontes \code{builtin}, \code{proc} e \code{prompt} contêm o código responsável pela
shell. O primeiro contém as funções que executam os comandos pré-existentes como enumerados na
seção anterior. O segundo trata da criação de novos processos, seja por meio de \code{rodeveja},
\code{rode} ou executando um comando diretamente. O último cuida da entrada e da atualização do
prompt.

O código em \code{main.c} apenas roda \code{prompt_readline} e \code{prompt_print} até que o
usuário peça que a shell seja terminada.

\section{Argumentos}

Iremos apresentar nesta seção a estrutura de dados \code{args_t} que contem uma lista de argumentos
que passaremos a cada função.

\begin{minted}[linenos=true,gobble=2,frame=lines]{c}
  typedef struct {
    /* List of string to be passed as arguments. */
    string_t **s;
    /* Argument count. */
    int c;
  } args_t;
\end{minted}

Cada lista de argumentos \code{args_t} contém um número \code{(args_t).c} de argumentos. Para
facilitar o uso de \code{args_t}, as seguintes funções de manipulação de \code{args_t} foram
criadas:

\begin{minted}[linenos=true,gobble=2,frame=lines]{c}
  /* Pops the first string_t from args_t *a. */
  args_t *args_pop(args_t *a);

  /* Pops the back of the list of arguments. */
  args_t *args_pop_back(args_t *a);

  /* Pushes argument s to the front of list of arguments a. */
  args_t *args_push(args_t *a, string_t *s);

  /* Pushes argument s to the back of list of arguments a. */
  args_t *args_push_back(args_t *a, string_t *s);
\end{minted}

Estas funções agem como se \code{args_t} fosse uma fila dupla. Iremos usar estas funções para
mandarmos a shell rodar um comando em \textit{background} ou \textit{foreground}.

\section{Comandos}

Nesta seção iremos descrever cada comando pedido no enunciado do EP, explicitando quais foram as
chamadas de sistema utilizados e como foram feitos.

O código para estes comandos estão em \code{builtin.c}. Todas estas funções contêm o prefixo
\code{builtin_} antes do nome do comando e seguem o seguinte protótipo:

\begin{minted}[gobble=2]{c}
  static int builtin_nomecomando(args_t *args);
\end{minted}

\subsection{\code{protegepracaramba}}

Este comando toma como argumento um arquivo \code{filename}. Em seguida,\\ \code{protegepracaramba}
chama a chamada de sistema \code{chmod} com parâmetro \code{000}. Se a função de sistema teve
sucesso, o arquivo \code{filename} terá agora permissão \code{000}. Senão, o comando
\code{protegepracaramba} retornará o valor \code{1} e irá imprimir, na saída padrão, uma mensagem
de erro dada pela chamada \code{chmod}. Se não houver erros, o comando retornará \code{0}.

\begin{minted}[gobble=2,frame=lines,linenos=true]{c}
  static int builtin_protegepracaramba(args_t *args) {
    if (args->c < 2)
      puts("Usage: protegepracaramba filename\n  "
           "Sets permissions to 000.");

    if (chmod(args->s[1]->str, 0000) < 0) {
      PRINT_ERR();
      return 1;
    }
    return 0;
  }
\end{minted}

\subsection{\code{liberageral}}

Assim como \code{protegepracaramba}, o comando \code{liberageral} toma um arquivo \code{filename} e
usa \code{chmod} para dar permissão \code{777} para \code{filename}. Se houver algum erro, o
comando irá imprimir a mensagem e número de erro na saída padrão e retornar \code{1}. Senão, apenas
retorna \code{0}.

\begin{minted}[gobble=2,frame=lines,linenos=true]{c}
  static int builtin_liberageral(args_t *args) {
    if (args->c < 2)
      puts("Usage: liberageral filename\n  "
           "Sets permissions to 777.");

    if (chmod(args->s[1]->str, 0777) < 0) {
      PRINT_ERR();
      return 1;
    }
    return 0;
  }
\end{minted}

\subsection{\code{rodeveja}}

O comando \code{rodeveja} aceita como argumentos um comando \code{command} e argumentos para
\code{command}.

\begin{lstlisting}[mathescape=true,style=nonumbers]
  $\$$ rodeveja command [arguments]
\end{lstlisting}

Onde \mcode{[$\cdot$]} indica uma lista de argumentos de tamanho arbitrário (e possivelmente zero).
Esta função irá rodar o comando dado como parâmetro e imprimir o resultado.

\begin{minted}[gobble=2,frame=lines,linenos=true]{c}
  static int builtin_rodeveja(args_t *args) {
    int res;

    if (args->c < 2)
      puts("Usage: rodeveja cmd [args]\n  "
           "Runs command cmd with arguments args "
           "and outputs its exit status.");

    args_pop(args);
    res = proc_exec(args);
    printf("=> programa '%s' retornou com codigo "
           "%d.\n", args->s[0]->str, res);
    return res;
  }
\end{minted}

Assim que passarmos \code{args_t *args}, teremos uma lista de argumentos cuja cabeça da lista é a
string \code{rodeveja}. Para passarmos o comando certo dado por \code{command [arguments]}, temos
de tirar o primeiro elemento da fila dupla \code{args}. Para isso usaremos \code{args_pop}.

Após eliminarmos a cabeça da lista, passamos o restante dos argumentos para a função
\code{proc_exec}, que veremos na próxima seção. Note que esta função não roda nenhuma chamada de
sistema explicitamente. Cabe a \code{proc_exec} fazer tais chamadas.

\subsection{\code{rode}}

O comando \code{rode} é semelhante a \code{rodeveja}, contendo apenas duas diferenças. Como o
executável dado como argumento será rodado em \textit{background}, iremos dizer a \code{proc_exec}
para não esperar o processo filho acabar para retornar o valor. Além disso, não escreveremos o
resultado do comando na saída padrão como fizemos em \code{rodeveja}.
\newpage
\begin{minted}[gobble=2,frame=lines,linenos=true]{c}
  static int builtin_rode(args_t *args) {
    if (args->c < 2)
      puts("Usage: rode cmd [args]\n  "
           "Runs command cmd with arguments args.");

    args_pop(args);
    args_add(args, copy_string("&", 2));
    return proc_exec(args);
  }
\end{minted}

Note que, além de chamarmos \code{args_pop(args)} como fizemos em \code{rodeveja}, também
adicionamos um novo argumento no final da fila \code{args}: a string \code{\"\&\"}. Essa string
diz a \code{proc_exec} para não esperar pelo processo filho.

\section{Rodando comandos externos}

Nesta seção analisaremos o módulo \code{proc}. Este módulo contém apenas uma função:

\begin{minted}[gobble=2]{c}
  int proc_exec(args_t *args);
\end{minted}

Seja uma lista de argumentos $A$, e $|A|=n$. Então a função \code{proc_exec} segue o seguinte
algoritmo:

\begin{enumerate}[label*=\arabic*.]
  \item Toma a string $A[0]$ como o comando a ser rodado.
  \item Toma a lista de strings $A[1..n-2]$ como argumentos para $A[0]$.
  \item Cria um clone do processo com \code{fork}.
  \item Se for processo filho:
  \begin{enumerate}[label*=\arabic*.]
    \item Roda $A[0]$ dados $A[1..n-2]$ com \code{execve}.
    \item Se deu erro, imprime erro.
    \item Termina o processo.
  \end{enumerate}
  \item Senão, é processo pai:
  \begin{enumerate}[label*=\arabic*.]
    \item Se $A[n-1] = "\&"$, estamos rodando em \textit{background}:
    \begin{enumerate}[label*=\arabic*.]
      \item Retorna o valor fixo de 0.
    \end{enumerate}
    \item Senão, temos de esperar pelo filho:
      \begin{enumerate}[label*=\arabic*.]
        \item Se houve erro, imprime erro e retorna o número do erro.
        \item Senão, retorna o valor de \mcode{execve($A[0..n-1]$)}.
      \end{enumerate}
  \end{enumerate}
\end{enumerate}

Passos 4 e 5 estarão rodando concorrentemente. A princípio poderíamos achar que isso pode trazer
algum problema, mas ao analisarmos é fácil ver que os dois eventos são independentes. Caso o filho
acabe antes, então o pai saber se deve esperar ou não não importará. Se o filho acabar depois, o
pai então deverá esperar pelo filho, e portanto ficará ``preso'' em 5.2.

Usamos nesta função duas chamadas de sistema: \code{fork} e \code{execve}. A primeira cria um clone
do nosso processo no passo 3. Se o resultado desta chamada for igual a zero, então estamos rodando
no processo filho e portanto entraremos no bloco do passo 4. Caso contrário estaremos no processo
pai e portanto entraremos no bloco do passo 5.

\section{Compilando}

Para compilar, basta rodar o Makefile em \code{/usr/local/src/mac422shell}. O Makefile usa o
\code{gcc}, que no Minix que usei coloquei em \code{/bin/}. O binário resultante é um executável
\code{mac422shell}, que deve ser equivalente ao já presente em \code{/usr/local/bin/}.

A shell \code{mac422shell} foi compilada com as seguintes flags:

\begin{lstlisting}[style=nonumbers]
  -Wall -ansi -pedantic
\end{lstlisting}

Não imprimindo nenhum erro durante a compilação.

\end{document}
