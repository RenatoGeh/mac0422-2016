\documentclass{amsart}

\usepackage[brazilian]{babel}
\usepackage[utf8]{inputenc}
\usepackage{graphicx}
\usepackage{mathtools}
\usepackage{amsthm}
\usepackage{amsfonts}
\usepackage{hyperref}
\usepackage[singlelinecheck=false]{caption}
\usepackage{enumitem}
\usepackage[justification=centering]{caption}
\usepackage{indentfirst}
\usepackage{listings}
\usepackage{minted}

\makeatletter
\def\subsection{\@startsection{subsection}{3}%
  \z@{.5\linespacing\@plus.7\linespacing}{.1\linespacing}%
  {\normalfont}}
\makeatother

\DeclareMathOperator*{\argmin}{arg\,min}
\DeclareMathOperator*{\argmax}{arg\,max}

\newcommand\defeq{\mathrel{\overset{\makebox[0pt]{\mbox{\normalfont\tiny\sffamily def}}}{=}}}

\captionsetup[table]{labelsep=space}

\theoremstyle{plain}

\newtheorem*{definition}{Definição}
\newtheorem{theorem}{Theorem}
\newtheorem{proposition}{Proposition}
\newtheorem{exercise}{Exercise}

\newcommand{\set}[1]{\mathcal{#1}}
\newcommand{\pr}{\mathbb{P}}
\renewcommand{\implies}{\Rightarrow}
\newcommand{\prompt}{$\$$}

\newcommand{\code}[1]{\lstinline[mathescape=true]{#1}}
\newcommand{\mcode}[1]{\lstinline[mathescape]!#1!}

\lstset{%
  language=C,
  numbers=left,
  breaklines=true,
  keywordstyle=\bfseries,
  basicstyle=\ttfamily
}

\lstdefinestyle{numbers}{numbers=left, stepnumber=1, numberstyle=\tiny, numbersep=10pt}
\lstdefinestyle{nonumbers}{numbers=none}

\setlength{\parskip}{1em}

\title[]{\rule{10.5cm}{0.8pt}\\Exercício-Programa 4
\\\vspace{2mm}\footnotesize
  Sistemas Operacionais --- MAC0422\\\rule{10cm}{0.8pt}}
\author[]{Renato Lui Geh\\NUSP\@: 8536030\\
          Guilherme Freire\\NUSP\@: 7557373}

\begin{document}
\date{\today}
\maketitle

\section{Introdução}

O EP foi feito em um Minix 3.1.2a simulado pela VM VirtualBox. Os arquivos fonte estão localizados
em \code{/usr/local/}.

Os arquivos modificados foram:

\begin{itemize}
  \item \code{/usr/local/include/minix/callnr.h}
  \item \code{/usr/local/src/include/minix/callnr.h}
  \item \code{/usr/local/src/lib/posix/Makefile.in}
  \item \code{/usr/local/src/servers/fs/inode.c}
  \item \code{/usr/local/src/servers/fs/inode.h}
  \item \code{/usr/local/src/servers/fs/proto.h}
  \item \code{/usr/local/src/servers/fs/table.c}
  \item \code{/usr/local/src/servers/fs/open.c}
\end{itemize}

As versões modificadas estão em \code{/usr/local/}, assim como os arquivos não modificados. Deste
jeito, pode-se rodar \code{/usr/local/src/tools/Makefile} sem alterar o código original. Um arquivo foi adicionado:

\begin{itemize}
  \item \code{/usr/local/src/lib/posix/_lsr.c}
\end{itemize}

Quando os blocos de código transcritos neste relatório não forem muito grandes, vamos indicar as
modificações feitas. Um símbolo \code{-} no início da linha indica a linha original no Minix. Um
símbolo \code{+} no início da linha indica a nova linha adaptada para o EP\@. Uma linha vazia com o
símbolo \code{-} indica que no código original a linha não existia. Analogamente, \code{+} em uma
linha vazia indica que deletamos a linha original. Um \code{#} indica um comentário no código, ou
seja, a linha indicada por este símbolo não existe no arquivo original.

\section{lsr()}

Para o \code{lsr}, nós fizemos uma system call como fizemos nos EPs anteriores. Em \code{inode.c},
mudamos a alocação do i-node para guardar uma lista de PIDs: \code{i_pids}. Alteramos todas as
ocorrências de alocação e liberação de memória para adequar-se ao nova entrada na \code{struct}.
Toda vez que o i-node é referenciado, nós chamamos a função \code{getpid} e guardamos o PID do
processo que usou o i-node. Quando o processo não usa mais o i-node chamando a função de liberação,
nós retiramos o PID da lista.

Em \code{open.c}, implementamos \code{do_lsr}, imprimindo a lista de PIDs e também imprimindo os
números dos blocos diretos, indiretos e duplamente indiretor.

Infelizmente, o código compila mas não roda. :(

\section{Arquivos imediatos}

Neste EP, essa função não foi implementada por dificuldades técnicas.

Inicialmente, procuramos materiais de referência na Internet para nos auxiliar com essa tarefa. Encontramos esse documento \emph{https://minixnitc.github.io/report.pdf} que descreve uma implementação muito próxima do nosso objetivo. Ele foca principalmente na leitura e escrita em arquivos imediatos e seu comportamento.

Procuramos entender o algoritmo descrito, porém ele omite partes importantes, de forma que não foi possível compreender os detalhes, além de ter estruturas que não eram necessárias para o nosso problema. Por esses motivos não foi possível utilizá-lo como guia.

O passo seguinte foi tentar uma implementação partindo do zero. Uma \emph{flag} para apontar que o i-node é um arquivo imediato foi criada no arquivo \code{inode.h} e inicializada no arquivo \code{inode.c}. Após isso, procuramos nos outros arquivos do FS onde utilizar essa \emph{flag}, porém não a estrutura de i-node é bastante confusa e não conseguimos implementar essa funcionalidade. Nenhuma tentativa foi bem sucedida, então resolvemos desfazer todas as alterações.


\section{Observações importantes quanto a execução do EP}

\emph{O EP não funciona}. Ao dar boot na imagem gerada, um erro ocorrerá. Não conseguimos encontrar o motivo pelo qual esse erro ocorre, mas deixamos as alterações feitas.

Caso ainda queira executar a versão gerada, é necessário levar em conta que, como usamos um Floppy Controller na nossa VM, a tela inicial irá indicar que não foi encontrado um
local de boot. Para resolver isto, pressione \code{F12} e em seguida pressione \code{1}. Isto
selecionará o controlador principal (a que possue a imagem do Minix) como local de boot.

Como não mudamos o caminho de boot padrão do Minix, quando a VM for rodada, deve-se dar boot na
imagem correta. Por padrão, a VM irá dar boot na imagem padrão original.

É recomendável que se recompile o Minix novamente para garantir que tudo esteja o mais recente
possível. Caso não se recompile o Minix, a imagem em \code{/boot/image} mais recente é:

\begin{lstlisting}[frame=leftline,mathescape=true,style=nonumbers]
/boot/image/3.1.2ar45
\end{lstlisting}

Para rodar a imagem escolhida, basta indicar o caminho. Por exemplo, caso a imagem desejada seja
\code{/boot/image/3.1.2ar45}, então:

\begin{lstlisting}[frame=leftline,mathescape=true,style=nonumbers]
# Garanta que esteja na tela de boot.
shutdown
# Indique qual imagem deve ser escolhida.
image=/boot/image/3.1.2ar45
# Faca o boot.
boot
\end{lstlisting}

\end{document}
